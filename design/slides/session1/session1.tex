\documentclass{beamer}
\usetheme[pageofpages=/,
          titleline=true,
          alternativetitlepage=true,% Use the fancy title page.
          ]{Torino}
\usecolortheme{metascale}
\usepackage{listings}
\ifxetex
  \usepackage{fontspec}
  \defaultfontfeatures{Mapping=tex-text}
  \setsansfont[ItalicFont={GillSansMTPro-BookItalic}]{GillSansMTPro-Book}
  \setmonofont{Inconsolata}
  \newcommand{\codefont}{\ttfamily}
\else
  \usepackage[utf8x]{inputenc}
  \usepackage[nott]{inconsolata}
  \newcommand{\codefont}{\fontfamily{fi4}\selectfont}
\fi
\usepackage{graphicx}
\usepackage{color}
\usepackage{multicol}
\usepackage{url}
\usepackage[frenchb]{babel}

\lstset{language=c++,
        basicstyle=\tiny,
        tabsize=2,
        showstringspaces=false,
        frame=leftline,
        columns=fullflexible,
        keywordstyle=\color[rgb]{0.64,0.13,0.11},
        commentstyle=\color[rgb]{0.27,0.27,0.28},
        stringstyle=\color[rgb]{0.82,0.16,0.14},
        columns=fixed
        }

\title{Software Design - Session 1}
\subtitle{Designing SOLID Softwares  - Part 1/2}
\author{Joel Falcou}
\institute{Laboratoire de Recherche en Informatique - Université Paris Sud 11}
\date{}

\pgfdeclareimage[interpolate=true,height=7cm]{solid}{../../../media/solid}
\pgfdeclareimage[interpolate=true,height=7cm]{single}{../../../media/solid-1}
\pgfdeclareimage[interpolate=true,height=7cm]{openclose}{../../../media/solid-2}

\begin{document}

\begin{frame}[plain]
\titlepage
\end{frame}
\section{Why Designing SOLID ?}
\frame
{
  \frametitle{Why Designing SOLID ?}
  \begin{center}\pgfuseimage{solid}\end{center}
}

\frame
{

}

\frame
{
  \frametitle{Additional Reference Materials}
}

\section{<S>ingle Resposability}
\frame
{
  \frametitle{The Single Responsability Principle}
  \begin{center}\pgfuseimage{single}\end{center}
}

\frame
{

}

\section{<O>pen/Closed}
\frame
{
  \frametitle{The Open/Closed Principle}
  \begin{block}{Robert Martin, 19xx :}
 \emph{"Software entities should be \textbf{open} for extension, but \textbf{closed} for modification; that is, such an entity can allow its behavior to be modified without altering its source code."}
  \end{block}

  \begin{block}{In More Details:}
  \begin{itemize}
  \footnotesize
  \item \textbf{Open For Extension:} It is possible to extend the behavior of the module as the requirements of the application change.
  \item \textbf{Closed For Modification:} Extending the behavior of the module does not result in the changing of the source code or binary code of the module itself.
\end{itemize}
  \end{block}

}

\frame
{
  \frametitle{The Open/Closed Principle}
  \begin{center}\pgfuseimage{openclose}\end{center}
}

\frame
{
  \frametitle{}
}

\end{document}


